\chapter{Introduction}
Discontinuous Galerkin (DG) methods are a particular type of finite-element method that is particularly well suited for convection dominated problems.  The name is derived from their close relationship to standard Galerkin finite-element methods.  However, in contrast to typical Galerkin finite-element methods no continuity requirements are imposed on the basis functions.  DG methods are a particularly attractive numerical method because they sit on a very solid mathematical background.  DG methods come equipped with all the standard finite-element method proofs of stability, consistency, and convergence at any discretization order.  This is a property that is not shared with finite-volume and similar unstructured grid numerical methods.  

DG methods are, in their most basic form, a finite-element method.  However, typical finite-element methods are continuous finite-element methods where the basis functions, which approximate the discrete solution, are continuous at the element interfaces.  Continuous finite-element methods traditionally have been applied to linear structural and thermal analysis problems that constitute purely elliptic operators, and hence continuous basis functions are appropriate.  DG methods employ basis functions that are discontinuous at the element interfaces, which makes DG methods naturally suitable for computing convection dominated problems.  DG discretizations are an ideal choice for convection dominated problems because the discontinuous basis functions allow for upwind flux calculations using approximate Riemann solvers.  Employing approximate Riemann solvers at the element interfaces is a strategy that is borrowed from finite-volume methods.  Thus DG can be thought of as a combination of traditional finite-element and finite-volume methods.  The blending of these methods is the result of simultaneously viewing the element as a control volume and as a domain over which interpolation functions (which are also known as basis functions) may be defined.  However, since the DG method is a finite-element method, the order of accuracy and number of unknowns are coupled.  DG methods attain high-order accuracy by adding additional basis functions within the elements, which results in additional degrees of freedom for increased orders of accuracy.  Alternatively, finite-volume and finite-difference methods reconstruct high-order data from neighboring elements, which does not increase the total number of degrees of freedom.  Therefore, finite-volume and finite-difference methods do not couple the order of accuracy with the number of degrees of freedom.  The coupling of the order of accuracy and number of unknowns within an element is a non-trivial property of DG methods, which affects many aspects of solver robustness and hence is a recurring theme throughout this work.   However, locating extra unknowns within the elements can be advantageous, provided that great care is taken in constructing and implementing these methods.     

As problem size increases, the efficient use of parallel computers becomes more important.  DG methods add resolution to a given problem via two approaches.  DG methods can add resolution by increasing the number of degrees of freedom within the element, which results in increased parallel efficiency over low-order methods for unstructured grids\cite{nastase07}.  By locating the DoFs within the element, higher computational density is achieved and proportionally less inter-element data communication is required.  This makes high-order DG methods an ideal candidate for large scale parallel computing.  Contrarily, while high-order finite-difference methods have been developed, these methods require the construction of extended interpolation stencils.  Extending the interpolation stencil can cause parallel scaling to degrade as the order of accuracy is increased.  This degradation of parallel efficiency is a result of the stencils of the grid points on partition boundaries relying on information from multiple data points on neighboring processors.  Reference \cite{nastase07} has shown that high-order DG methods have the opposite trend, as the order of accuracy increases the parallel scalability increases as well.      