\documentclass[letterpaper]{article}
\usepackage{fullpage}
\usepackage{amssymb}
\usepackage{amsfonts}
\usepackage{amsmath}
\usepackage{stmaryrd}
\usepackage{graphicx}
\usepackage{subfigure}
\usepackage{pslatex}   % Good fonts psType 1 or \usepackage{mathptmx}
\usepackage{varioref}  % smart page, figure, table, and equation referencing
\usepackage{algorithm}
\usepackage{algorithmic}
%\usepackage{algorithm2e}
\usepackage{cancel}
\usepackage[small]{caption}
\usepackage{cancel}
\usepackage{tikz} 
\usetikzlibrary{shapes,snakes}
\usepackage{pgf}
\usepackage{listings}

%\usepackage{fancy headings}

  \newcommand{\eqnref}[1]{Eq. (\ref{#1})}                % Eq. (no)
  \newcommand{\figref}[1]{Figure \ref{#1}}                % Figure (no)
  \newcommand{\tblref}[1]{Table \ref{#1}}                % Table (no)
  \newcommand{\secref}[1]{Section \ref{#1}}                % Section (no)
  \newcommand{\incfig}{\centering\includegraphics*}  % Centered 
  \newcommand{\comment}[1]{\COMMENT{ \textcolor{blue}{ #1} }  }
%Constants
\newcommand{\half}{\frac{1}{2}}
%DG integrals
\newcommand{\pd}[2]{\frac{ \partial #1}{\partial #2}}
\newcommand{\volint}[1]{ \sum_{e \in \mathcal{T}_{h}}\int_{\Omega_k} #1 d\Omega_{e} }  
\newcommand{\surfint}[1]{\sum_{i\in \mathcal{I}_{h}}\int_{\Gamma^{i}} #1 ds}
\newcommand{\bsurfint}[1]{\sum_{b\in \mathcal{B}_{h}}\int_{\Gamma^{b}} #1 ds}
% Operators
\newcommand{\frechd}[2]{ #1'_{\left[ #2 \right]} }%\left( #3\right)} 
\newcommand{\diver}[1]{\nabla \cdot #1}
\newcommand{\abs}[1]{\left \lvert #1 \right \rvert}
\newcommand{\avg}[1]{\left\{ #1 \right\} }
\newcommand{\jump}[1]{\llbracket #1 \rrbracket}
\newcommand{\mat}[1]{\left[ #1 \right]}
\newcommand{\paren}[1]{\left( #1 \right)}
\newcommand{\sparen}[1]{\left[ #1 \right]}
\newcommand{\twonorm}[1]{\parallel #1 \parallel_{2}}
%variables
\newcommand{\hb}[1]{ {\bf #1}_{h} } % \hb for discrete symbol _{h} and bolded for vector in fields
% Fluxes 
\newcommand{\fc}{\vec{{\bf F}}_{c} \paren{\hb{u}}   }
\newcommand{\fv}{\vec{{\bf F}}_{v} \paren{\hb{u},\nabla \hb{u}} }
\newcommand{\fav}{\vec{{\bf F}}_{ad} \paren{ \epsilon,\hb{u},\nabla \hb{u} } }
\newcommand{\ec}{\vec{{\bf E}}_{c} \paren{\hb{u}}   }
\newcommand{\ev}{\vec{{\bf E}}_{v} \paren{\hb{u},\nabla \hb{u}} }
\newcommand{\eav}{\vec{{\bf E}}_{ad} \paren{ \epsilon,\hb{u},\nabla \hb{u} } }

\newcommand{\hc}{\mathcal{H}_{c} \paren{ \hb{u}^{+},\hb{u}^{-},\vec{n} } }  
\newcommand{\hv}{\mathcal{H}_{v} \paren{ \hb{u},\hb{u}^{-},\phi_{i}^{+}, \phi_{i}^{-},\nabla \hb{u}^{+},\nabla \hb{u}^{-},
\vec{n} } }
\newcommand{\hav}{\mathcal{H}_{ad}\paren{ \epsilon^{+},\epsilon^{-},\hb{u}^{+},\hb{w}^{+}, \hb{w}^{-},\hb{u}^{-},\nabla \hb{u}^{+},\nabla \hb{u}^{-},\vec{n} } }
\newcommand{\hcb}{\mathcal{H}_{c}^{b} \paren{ \hb{u}^{b} \paren{ \hb{u}^{+} },\vec{n} } } 
\newcommand{\hvb}{\mathcal{H}_{v}^{b} \paren{ \hb{u}^{b} \paren{\hb{u}^{+}},\phi_{i}^{+},\nabla \hb{u}^{+},\vec{n} } }
\newcommand{\havb}{\mathcal{H}_{ad}^{b} \paren{ \epsilon^{+}, \hb{u}^{b}\paren{ \hb{u}^{+}},{\bf w}^{+}, \nabla \hb{u}^{+},\vec{n} } }
%Non-linear variables
\newcommand{\Resid}[1] {{\bf R}(#1)}
\newcommand{\Residp}[1] {{\bf R}_{p}(#1)}
%FIgures
\newcommand{\figwidth}{.48\textwidth}
\newcommand{\lfigwidth}{.68\textwidth} %small version .68, large version .75
%%% Footer
%\lfoot[\fancyplain{}{}]{Approved for public release \\ distribution unlimited.}
\newcommand{\createav}{CREATE\textsuperscript{TM}-AV }

%%%%% Listings environments
\lstdefinestyle{Meshfile}
{
sensitive=false,
keywordstyle=\color{blue},
morekeywords={Nodes, Elements, BcFaces, BcID, Connectivity, Nodal, Boundary, Face, IDs},
basicstyle=\ttfamily,
numbers=left, 
numberstyle=\ttfamily\color{gray}\small, 
numbersep=5pt, 
frame=single, 
rulecolor=\color{black}, 
framexrightmargin=5pt,
backgroundcolor=\color{white}, 
}


\title{Control Volume Finite Difference: PEBI Grid Implementation}
\author{Nicholas K. Burgess}
\begin{document}
\maketitle
\section{Introduction}
Control volume finite difference (CVFD) methods meshes are a popular method for discretizing the question govern the multi-phase flow through porous media.  CVFD methods require a type of unstructured mesh known as a perpendicular edge bisection (PEBI) mesh.  In this work the origins of why PEBI meshes are required as well as the generality of the meshes is explored.  It will be shown that for the single phase pressure equation CVFD methods are equivalent to Galerkin finite element (GFEM) methods under certain geometrical constraints.  The analysis of this work is restricted to two spatial dimensions in order to simplify the diagrams and terms and number of terms.    
\section{Perpendicular Edge Bisection (PEBI) Meshes}
Perpendicular edge bisection (PEBI) meshes are generated by connecting circumcirles of neighboring triangles.  
\subsection{Circumcircles of Triangles}
The circumcircle of a triangle is the circle that passes through all three nodes of a triangle.  The center of the circumcircle is denoted the circumcenter.  \figref{fig:tri_circumcenter} shows two example triangles and their circumcircles and circumcenters.  
\begin{figure}[h!]
\centering
\subfigure[Triangle with circumcenter inside the triangle]{\includegraphics[width=\figwidth]{tri_circumcenter_inside.pdf}}
\subfigure[Triangle with circumcenter outside the triangle]{\includegraphics[width=\figwidth]{tri_circumcenter_outside.pdf}}
\caption{Example of triangle circumcenters.}
\label{fig:tri_circumcenter}
\end{figure}
The equations covering the the circumcenter coordinates ($x_{c}, y_{c}$) and circumcircle ($r_{c}$) radius are given as 
\begin{equation}
\paren{x_{i} - x_{c}}^{2} + \paren{y_{i} - y_{c}}^{2} = r_{c}^{2} \quad \forall i \in [1,3]
\end{equation}
Solving these equations gives the location of the circum cirlce as
\begin{equation}
\begin{split}
& x_{c} = \frac{ r_{1}\paren{y_{3} - y_{1}} - r_{2}\paren{y_{2} - y_{1}} }{\paren{x_{2} - x_{1}}\paren{y_{3} - y_{1}} - \paren{x_{3} - x_{1}}\paren{y_{2} - y_{1}}} \\
& x_{c} = \frac{ r_{2}\paren{x_{2} - x_{1}} - r_{1}\paren{x_{3} - x_{1}} }{\paren{x_{2} - x_{1}}\paren{y_{3} - y_{1}} - \paren{x_{3} - x_{1}}\paren{y_{2} - y_{1}}} \\
& r_{1} = \half\paren{x_{2}^{2} - x_{1}^2 + y_{2}^{2} - y_{1}^{2}} \\
& r_{2} = \half\paren{x_{3}^{2} - x_{1}^2 + y_{3}^{2} - y_{1}^{2}} \\ 
\end{split}
\end{equation}
Connecting the cirumcenters of neighboring in anti-clockwise order gives a cell (sometimes denoted in the literature as a Voronoi cell) that surrounds a given grid node.  

Consider a domain $\Omega$ with boundary $\partial \Omega = \Gamma$ and let this domain be triangulated as $\mathcal{T}_{h} = \avg{ \Omega_{k} \vert \bigcup_{k} \Omega_{k} = \Omega_{h}}$ where $\Omega_{h}$ is $\Omega$ with a piece-wise linear boundary $\Gamma_{h}$.  
\figref{fig:PEBI_mesh}
\begin{figure}[h!]
\centering
\includegraphics[width=\figwidth]{PEBI_Mesh.pdf}
\caption{Example PEBI mesh of $\Omega$.  Black lines denote triangle edges and red lines denote PEBI edges.}
\label{fig:PEBI_mesh}
\end{figure}

Let $PE_{{h}} $ denote the set of all edges connecting connecting the circum 
One can also imagine that a PEBI mesh can be created from quadrilaterals, however there are significant geometric restrictions on when a PEBI mesh of quandrilaterals is obtainable.   
\subsection{Discretizing the Flow Equations on a PEBI Mesh}
For the sake of simplicity this work will consider the equations governing steady state single-phase flow through porous media.
\begin{equation}
\begin{split}
&\nabla \cdot \paren{\vec{u}} = S\paren{u} \\
& \vec{u} = -\frac{\sparen{K}}{\mu} \nabla P
\end{split}
\end{equation}  
Inserting the Darcy Law for the velocity into the mass conversation equation gives 
\begin{equation}
-\nabla \cdot \paren{ \frac{\sparen{K}}{\mu} \nabla P} =  S\paren{u}
\end{equation}
\end{document}